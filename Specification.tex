\sekshun{Specification}
\label{Specification}
\index{specification}

\section{Overview of the Genetic Algorithm}
\label{Genetic_Algorithm_Overview}
\index{genetic algorithm}
\index{genetic algorithm!specification}

A flowchart representing the operation of the genetic algorithm used in the program is shown in 
Figure~\ref{fig:flowchart}. The algorithm follows Johnston.\cite{johnston} Specific features are described below.

\begin{figure}
\begin{flowchart}
\node (start) [startstop] {Start};
\node (generate_initial_population) [process, right of=start, xshift=2cm] {Generate Initial Population};
\draw [arrow] (start) -- (generate_initial_population);
\node (relaxation) [process, right of=generate_initial_population, xshift=2cm] {Relaxation};
\draw [arrow] (generate_initial_population) -- (relaxation);
\node (select_parents_for_crossover) [process, below of=relaxation] {Select Parents for Crossover};
\node (assign_fitness) [process, right of=select_parents_for_crossover, xshift=2cm] {Assign Fitness};
\draw [arrow] (relaxation) -| (assign_fitness);
\draw [arrow] (assign_fitness) -- (select_parents_for_crossover);
\node (crossover) [process, below of=select_parents_for_crossover] {Crossover};
\draw [arrow] (select_parents_for_crossover) -- (crossover);
\node (mutation) [process, below of=crossover] {Mutation};
\draw [arrow] (crossover) -- (mutation);
\node (remove_twins) [process, below of=mutation] {Remove Twins};
\draw [arrow] (mutation) -- (remove_twins);
\node (offspring) [io, left of=crossover, xshift=-2cm] {Offspring};
\node (mutants) [io, left of=remove_twins, xshift=-2cm] {Mutants};
\node (select_generation) [process, left of=mutants, xshift=-2cm] {Select Generation};
\node (next_generation) [io, below of=mutants, yshift=-0.5cm] {Next Generation};
\node (converged) [decision, right of=next_generation, xshift=2cm] {Converged?};
\node (final_population) [io, below of=converged, yshift=-1cm] {Final Population};
\node (stop) [startstop, left of=final_population, xshift=-2cm] {Stop};
\node (previous_generation) [io, right of=final_population, xshift=2cm, yshift=-2cm] {Previous Generation};
\draw [arrow] (crossover) -- (offspring);
\draw [arrow] (offspring) -| (select_generation);
\draw [arrow] (remove_twins) -- (mutants);
\draw [arrow] (mutants) -- (select_generation);
\draw [arrow] (select_generation) |- (next_generation);
\draw [arrow] (next_generation) -- (converged);
\draw [arrow] (converged) -| node[pos=0.1,anchor=north] {no} (assign_fitness);
\draw [arrow] (converged) -- node[anchor=west] {yes} (final_population);
\draw [arrow] (converged) -| (previous_generation);
\draw [arrow] (previous_generation) -| ($(select_generation.west)-(0.5,0)$) -- (select_generation);
\draw [arrow] (final_population) -- (stop);
\end{flowchart}
\caption{Genetic algorithm flowchart}
\label{fig:flowchart}
\end{figure}

\subsection{Generate Initial Population}

For a given cluster size (nuclearity $N$), an initial population of $N_{pop}$ clusters ($10<N_{pop}<30$) are generated
randomly. The $x$, $y$, and $z$ coordinates are chosen randomly in the range $[0,N^{1/3}]$, ensuring that the cluster
size scales linearly with $N$.

\subsection{Write Cartesian Coordinates of a Random Cluster to File}

  \begin{enumspec}
  \item\spec{1} Test \chpl{randomClusterCartesianCoordinatesTest}: 
    generates random cartesian coordinates for a cluster and writes
    them to a file.\\
    \begin{tabular}{r r p{6cm}} \toprule
      \textbf{output} x 4  & \chpl{stdout: true}   & test passed \\ 
                           & \chpl{stdout: false}  & test failed \\ \midrule
      \textbf{modules loaded} & \multicolumn{2}{l}{\chpl{randomClusterCartesianCoordinates}} \\ \midrule
      \textbf{requirements met} & \multicolumn{2}{l}{\meetsreq{1,1.1}} \\ \bottomrule
  \end{tabular}
  \end{enumspec}

\begin{chapelexample}{randomClusterCartesianCoordinatesTest.chpl}
  A test for \lstinline{randomClusterCartesianCoordinates}. 
  \begin{chapelpre}
  \end{chapelpre}
  \begin{chapel}
use randomClusterCartesianCoordinates;
use Random;
  
//set seed
var atoms:string = ["U", "U", "O", "O"];
var numberOfClusters:int = 4; 
var minBondArray: real; 
var maxDiff: real = 0.0001;

for clusterIndex in 1..numberOfClusters {
  clusterFilename = "cluster_" + clusterIndex + ".xyz";
  writeln("Writing cluster to file: ",clusterFilename);
  var clusterFile = open(clusterFilename,iomode.cw);
  var writer = clusterFile.writer();

  randomClusterCartesianCoordinates(
    atoms,
    minBondArray,
    clusterIndex,
    writer);

    //read each file back in

    //compare file contents to expected
  //writeln((abs(fileContents - expected) <= maxErr));

  writer.flush(); 
  writer.close();
}
\end{chapel}
\begin{chapelpost}
\end{chapelpost}
\begin{chapeloutput}
true
true
true
true
\end{chapeloutput}
\end{chapelexample}

